\newpage
\section{高阶}
\setcounter{subsection}{40}
\subsection{关注面子的人越来越多了}
The global cosmetics industry makes a ton of money. That includes shampoo, makeup, perfume, cologne, deodorant, soap. The list goes on. Analysts expect the industry's revenue to grow within the next 4 years to more than \$379 billion.

But when you break down that number, it's skin care that's driving much of the growth. And it's not slowing down.

We've interviewed three people about the most expensive skin care items they've used:

“I've used moisturizers upwards in like 250-dollar range.”

“The most expensive single item would be around 50 dollars. I try not to spend more than 50 for like an essence or moisturizer.” 

“Probably like this really expensive moisturizer that I kind of got conned into buying for about 150 dollars.”

Skin care is a big moneymaker for big beauty brands. Over the last 5 years, skin care has grown so much that it's become the largest piece of the pie when you break apart the beauty industry by product category.

What's really driving a lot of that performance is actually wellness and health, and really natural brands. That's what's really driving the performance of skincare.

Wellness doesn't mean just being healthy. In 2019, it means clean eating, the latest fitness fads and the no-makeup look, which actually does involve some makeup. At any rate, the wellness trend is driving consumers to take better care of their skin.

\subsection{塑料袋其实比纸袋环保}
Plastic bags seem to be one of the most hated things on the planet. But did you know that when it comes to things like climate change, paper and cotton bags are actually much worse for the planet than plastic? In fact, the plastic bag was originally invented to help save the planet.

Sten Gustaf Thulin is credited with inventing the plastic bag we know today back in Sweden in 1959. Sten had an idea. He worked out if he could create a strong bag that was light and would last for ages, then people could use it over and over again. And that would mean fewer trees being chopped down, which would be better for the environment.

Problem is, these bags were so convenient that we humans got lazy. We just throw them away after we were done with them.

Paper carrier bags ... they use more energy, they use a lot of water to produce and they're heavier as well. So, depending on where they're made, there's an extra environmental impact of transporting them to the shops. Cotton bags are even worse. Cotton uses a lot of water and it is quite an intensive crop to produce.

To be as environmentally friendly as a single use plastic bag that's getting recycled, a paper bag has to be used at least 3 times. While cotton bags need to be used at least 131 times to have the same environmental impact as a single use plastic bag which gets recycled.

\subsection{机器人时代的来临}
By 2020, Japan plans on integrating more robots into society than anywhere else on earth, but what does that mean for us? Are they going to replace humans or enhance them? Or just be non-stop dance partners?

Erika is one of the most human-like robots on earth. Her creator sees a future where robots like her will replace people in certain jobs. She may not be able to walk because right now her legs do not work.

OK, let's not dwell on the downsides. According to Dr. Hiroshi Ishiguro, she'd be a perfect fit for specific roles, like a receptionist, because she can communicate but never needs a lunch break.

OK, so there's still plenty of work out there for us, or is there? Here in Sasebo humans are being squeezed out at this restaurant which is testing out robots in all sorts of roles.

The government is investing more than half a billion dollars nationwide into its project, the Robot Revolution Initiative. So, will all these new robots actually change our everyday lives? If anyone would know, it's Dr. Masahiro Mori\sn{森政弘}.

After inventing a few robots of his own, he spent decades on groundbreaking research. He's most famous for a theory called the “uncanny valley”, which explains why human-like robots can seem creepy.

OK, doctor, so after dedicating your life to studying robots, will this robot revolution be a good thing or a bad one?
To decide whether it's good or bad, you can kill someone with a knife, right? However, you can also do a simple operation with this tool too. And in that way, it can save somebody's life.

So, whether robots can kill people or save people is up to us. Well, that's encouraging.
\subsection{人脸识别}
From China to the United States, facial recognition technology is being used and tested.

For surveillance, you can catch the face in public and find what you want to find; and for commercial use, you can find VIPs when they come to the store, so you can provide special services for them.

Great advances have been made in facial recognition technology in recent years.
Machines can now match faces even when the lighting isn't good or when a full shot of the face is not available. A side shot or moving image of the face may be enough for artificial intelligence to make a match.

“Where this is going is very exciting. Think about everyday items that we have that are going away. Our house keys, our car keys, our ATM cards, our passwords are all starting to go away and instead, we're going to be using facial recognition. Smartphones, of course now are using facial recognition. Laptops have facial recognition on them.”

The ability to capture an image of a person without consent comes with privacy concerns.
The biggest concern in the US is not necessarily about the government. It's the big companies where there are really no limits on how they can share data, what they can use it for and how they can exploit it.

Facial recognition researchers say a social framework should be created to guide the use of this technology so it can be used safely to benefit society and not exploit it.


\subsection{万物相连}
It sounds appealing while barely lifting a finger, you set your thermostat, start your coffeemaker, turn on the lights, fire up your favorite playlist. But what if the price of that convenience is your private information?

All of those connected gadgets carrying out all these useful jobs are part of what's become known as the Internet of Things.

And their increased prevalence in everyday life is forcing everyone to consider a fundamental give and take: comfort or privacy?

Tech companies, wireless carriers and all manner of startups are racing to connect whatever they can. And the benefits have been self-evident.

Smart speakers can answer questions, order groceries or book a reservation. Electronic monitors can let patients leave hospitals sooner or allow seniors to live at home for longer.

Looking forward, the worldwide adoption of 5G mobile technology will allow more IOT devices to talk to each other without human intervention at previously unreachable speeds.

That means homes that look after themselves and cars that take over the driving. As far as industrial applications, think smart factories and warehouses that can fulfill their own orders or notify supervisors about problems.

And yet this promise comes with potential downsides for the customers who own these devices. There's security or the lack thereof. Even if individual IOT devices are secure, more devices mean more vulnerabilities.

In one such example, hackers access the digital thermometer in a casinos aquarium and work their way from there through the casinos network to gain access to its database of high rollers.

Things start to get scarier when you imagine malware infecting a self-driving car or a surgical medical device.

There's also the question of utility. Does your baby really need a smart diaper? Does your pet need a smart door? Does the function of all these IOT devices make up for the increased electronic waste that they create?

“I don't understand.”
\subsection{大兴国际机场}
China is no stranger to record-breaking infrastructure projects. Already home to the world's largest dam, longest ocean crossing and most extensive high-speed rail network, the country has now completed one of the largest airports ever conceived.

Since China is one of the world's largest economies, the demand for air travel in and out
of China is extreme.

As the nation's centre of political power and as the second-largest city in a country of more than 1.3 billion people, the pressure on Beijing is particularly intense.

To alleviate pressure and safeguard Beijing's economic growth, a new hub was conceived.

With initial proposals suggesting 9 runways and capacity for 200 million passengers each year, the vast Daxing International was originally intended to replace Beijing Capital as the city's main airport.

Initially constructed with 4 runways and capacity for 72 million passengers each year, Daxing International can be expanded to 7 runways and could serve up to 100 million passengers annually when fully developed.

Plans for the new airport were released by Zaha Hadid\sn{扎哈·哈迪德(1950~2016)伊拉克裔英国女建筑师。} Architects in early 2015. Collaborating with airport specialists ADPI\sn{法国巴黎机场集团建筑设计公司} on the main terminal building, Hadid moved away from the linear model of airport design which often created sprawling facilities and instead introduced a six-pointed star arrangement that puts passengers less than 600 metres, around an eight-minute walk, from any departure gate.

While one arm of the building operates as an administrative centre, the remaining five accommodate the airport's 79 departure gates.
\subsection{台风海贝斯}\sn{Typhoon Hagibis 台风海贝斯,2019年太平洋台风季第19个被命名的风暴}
At least 18 people are confirmed to have died in Japan and a dozen more are missing after one of the most powerful storms there in decades. Typhoon Hagibis has now moved back into the Pacific after bringing record amounts of rainfall to large areas of the country.

Rivers are swollen and homes have been buried under landslides. The Japanese government has deployed tens of thousands of soldiers who try to help the rescue effort.

This is the town of Chikuma\sn{千曲市,位于日本长野县北部}in the Japanese Alps\sn{日本阿尔卑斯山脉,是日本本州中部山脉}, much of it now under water. Last night the levees here burst and the brown floodwaters came rushing in.

This town is far from the coast, people here were not expecting anything like this and many had not heeded warnings to evacuate. This morning, they took to their roofs and balconies as military helicopters hovered overhead.

It has been a race against time to get them out as floodwaters threatened to sweep away some of the homes.

Nearby a row of shiny bullet trains stands stranded amid the floodwaters, tens of millions of dollars in damage here alone. 

The water came up higher than my head in the house, it turned over all the furniture inside. It's like a washing machine now.

The scale of this storm has been astonishing. The area affected stretches for more than a thousand kilometers, the same distance as Cornwall to Edinburgh.
\subsection{喵星人}
They're cute, they're lovable, and judging by the 26 billion views of over 2 million YouTube videos of them pouncing, bouncing, climbing, cramming, stalking, clawing, chattering, and purring, one thing is certain: cats are very entertaining.

These somewhat strange feline behaviors, both amusing and baffling, leave many of us asking, "Why do cats do that?”

Throughout time, cats were simultaneously solitary predators of smaller animals and prey for larger carnivores. While the feline actions of your house cat Grizmo might seem perplexing, in the wild, these same behaviors, naturally bred into cats for millions of years, would make Grizmo a super cat.

Enabled by their unique muscular structure and keen balancing abilities, cats climbed to high vantage points to survey their territory and spot prey in the wild.

As wild predators, cats are opportunistic and hunt whenever prey is available. Since most cat prey are small, cats in the wild needed to eat many times each day, and use a ‘stalk, pounce, kill, eat strategy' to stay fed. This is why Grizmo prefers to chase and pounce on little toys and eat small meals over the course of the day and night.

In the wild, cats needed sharp claws for climbing, hunting, and self-defense. Sharpening their claws on nearby surfaces kept them conditioned and ready, helped stretch their back and leg muscles, and relieve some stress, too.

\subsection{全球贫困分类}
Poverty has become a critical issue in today's world. It concerns not only us sociologists, but also economists, politicians and business people. Poverty has been understood in many different ways. One useful way is to distinguish between three degrees of poverty—extreme poverty, moderate poverty, and relative poverty.

The first type of poverty is extreme poverty. It's also called absolute poverty. In extreme poverty, households cannot meet basic needs for survival. People are chronically hungry. They are unable to access safe drinking water, let alone health care. They cannot afford education for their children. In short, people who live in extreme poverty do not have even the minimum resources to support themselves and their families. Where does extreme poverty occur? Well, you can find it only in developing countries.

Well, what about moderate poverty? Unlike extreme poverty, moderate poverty generally refers to conditions of life in which basic needs are met, but barely. People living in moderate poverty have the resources to keep themselves alive, but only at a very basic level. For example, they may have access to drinking water but not clean, safe drinking water. They may have a home to shelter themselves but it does not have power supply, a telephone or plumbing.

The third kind of poverty is relative poverty. Relative poverty is generally considered to be a household income level which is below a given proportion of average family income. The relatively poor live in high-income countries but they do not have a high income themselves.

The method of calculating the poverty line is different from country to country but we can say that basically a family living in relative poverty has less than a percentage of the average family income.

For example, in the United States, a family can be considered poor if their income is less than 50 percent of the national average family income. They can meet their basic needs but they lack access to cultural goods, entertainment, and recreation. They also do not have access to quality health care or other prerequisites for upward social mobility\sn{upward social mobility:向上社会流动,可指个体在社会中获得财富或声望等。}.

Well, I have briefly explained to you how poverty can be distinguished as extreme poverty, moderate poverty, and relative poverty. We should keep these distinctions in mind when we research people's living conditions either in the developing or the developed world.

\subsection{好莱坞的包容危机}
Researchers have found not just a diversity problem in Hollywood, but actually an inclusion crisis.

With less than a week before an Oscars ceremony that has already been criticized for an all-white list of acting nominees, a study shows the film industry does worse than television.

Just 3.4 percent of film directors were female, and only 7 percent of films had a cast whose balance of race and ethnicity reflected the country's diversity.

When researchers looked at all TV shows, they also found that women of color over 40 were regarded as "largely invisible" and just 22 percent of TV series creators were female.

Overall, the study found half the films and TV shows had no Asian speaking characters and more than one-fifth of them had no black characters with dialogue.

The film industry still functions as a straight, white, boy's club.

When looking at how women are depicted, the study found female characters were four times more likely to be shown in sexy clothing, and nearly four times as likely to be referred to as physically attractive.

Across TV and film, the underrepresentation of non-white characters falls mostly on Hispanics.

Among more than 10,000 characters, proportions of white, black and Asian characters came close to U. S. population figures. But Hispanics were just 5.8 percent of characters, despite being about 17 percent of the U. S. population.