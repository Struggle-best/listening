\newpage
\section{中阶}
\setcounter{subsection}{18}
\subsection{危险的工作环境}
\begin{margintable}\vspace{-2cm}\footnotesize
	\includemedia[
	addresource=sound/fivestep/L19-危险的工作环境-挑战原声.mp3]{\faBullhorn}{VPlayer.swf}
\end{margintable}
Since a union representative visited our company to inform
us about our rights and protections, my coworkers
have been worrying about health conditions and complaining
about safety hazards in the workplace.

Several of the employees in the computer department,
for example, claim to be developing vision problems from
having to stare at a video display terminal for about 7
hours a day.

The supervisor of the laboratory is beginning to get headaches
and dizzy spells because she says it's dangerous to
breathe some of the chemical smoke there. An X-rays
technician is refusing to do her job until the firm agrees to
replace its outdated equipment.

She insists that it's exposing workers to unnecessarily high
doses of radiation. She thinks that she may have to contact
the Occupational Safety and Health Administration
and asked that government agency to inspect the department.

I've heard that at a factory in the area two pregnant
women who were working with paint requested a transfer
to a safer department, because they wanted to prevent
damage to their unborn babies. The supervisor of personnel
refused the request.

In another firm, the workers were constantly complaining
about the malfunctioning heating system, but the
owner was too busy or too mean to do anything about it.

Finally, they all met and agreed to wear ski-clothing to
work the next day. The owner was too embarrassed to talk
to his employees. But he had the heating system replaced
right away.
\subsection{大家都睡不够}
\begin{margintable}\vspace{-2cm}\footnotesize
	\includemedia[
	addresource=sound/fivestep/L20-大家都睡不够-挑战原声.mp3]{\faBullhorn}{VPlayer.swf}
\end{margintable}
The massive decline in sleep happened so slowly and quietly
that few seemed to notice the trend. Was it because
of the growing attraction of the Internet, video games and
endless TV channels? Never disconnecting from work?

No matter how it happened, millions of Americans are
putting their health, quality of life and even length of life in
danger.

New evidence shows why getting enough sleep is
a top priority. Some 40 percent of Americans get less than
seven hours of shut-eye on weeknights. “The link between
sleep and health, and bad sleep and disease, is becoming
clearer and clearer,” says Lawrence Epstein, a sleep
expert at Harvard University.

For example, sleep duration has declined from some
eight hours in the 1950s to seven in recent years. At the
same time, high blood pressure has become an increasing
problem. Blood pressure and heart rate are typically at
their lowest levels during sleep. People who sleep less
tend to have higher blood pressure, heart attack, diabetes,
weight gain and other problems.

Sleeping better may help fight off illness. “When
people are sleep-deprived, there are higher levels of
stress hormones in their bodies, which can decrease immune
function,” says Dr. Felice of Northwestern University
in Chicago.

A University of Chicago study shows people who sleep
well live longer. So say good night sooner, and it may help
you stay active and vital to a ripe old age.
\subsection{超市墙上没有表}
\begin{margintable}\vspace{-2cm}\footnotesize
	\includemedia[
	addresource=sound/fivestep/L21超市墙上没有钟表-挑战原声.mp3]{\faBullhorn}{VPlayer.swf}
\end{margintable}
A typical large supermarket offers around 17 000 to 20 000
items for sale and it wants to make sure that its customers
see as many of them as possible. That's why you'll
only find essential goods like bread, vegetables and meat
in completely different parts of the store.

Products with a high profit margin are always placed
on shelves within easy reach of the customer, while lower
margin items like sugar or flour are on the top or bottom
shelves.

Many people make shopping lists before they visit
supermarkets. But even so, around 60\% of all supermarket
purchases are the result of decisions that are made in the
store. For this reason, supermarkets try to attract their customers
by placing certain kinds of products next to each
other.

In the UK, beer will often be found next to items for babies
because research shows that fathers of babies buy these
items on their way home from work and will buy beer at
the same time. Research has also shown that this kind of
impulse buy happens more frequently when no sales assistants
are nearby.

Supermarkets have made selling such a fine art that their
customers often lose all sense of time. When interviewed,
customers normally guess they've only spent half an
hour in the supermarket even when they have been there
for over 45 minutes. But that shouldn't be too surprising.
Any witty profitable supermarket knows that it should keep
its clocks well hidden.

\subsection{电子书是未来}
\begin{margintable}\vspace{-2cm}\footnotesize
	\includemedia[
	addresource=sound/fivestep/L22-电子书-挑战原声.mp3]{\faBullhorn}{VPlayer.swf}
\end{margintable}
Digital textbooks are transforming the way many students
learn. All the Fairfax County\sn{费尔法克斯县,位于美维吉尼
	亚州北部} Public Schools have begun using online course
material for their middle and high school students.

This school year, the schools shifted from hard cover
to electronic textbooks for social studies in its middle and
high schools. The switch came after digital books were
used in 15 schools last year.


“Our students came to us technologically ready to use resources
from a variety of different places,” says Assistant
Superintendent Peter Noonan. “The world is changing constantly.
The online textbooks can change right along with
the events that are happening.”

There's a significant financial benefit as well. “Usually it is
between \$50 and \$70 to buy a textbook for each student,”
Noonan says, “which adds up to roughly \$8 million for
all of our students. We actually have purchased all of the
online textbooks for our students for just under \$6 million.”

\subsection{老话有理}
\begin{margintable}\vspace{-2cm}\footnotesize
	\includemedia[
	addresource=sound/fivestep/L23-老话有理-挑战原声.mp3]{\faBullhorn}{VPlayer.swf}
\end{margintable}
Proverbs, sometimes called sayings, are examples of folk
wisdom. They are little lessons which older people of a
culture pass down to the younger people to teach them
about life.

Many proverbs remind people of the values that are
important in the culture. Values teach people how to act,
what is right, and what is wrong. Because the values of
each culture are different, understanding the values of
another culture helps explain how people think and act.

Understanding your own cultural values is important too.
If you can accept that people from other cultures act according
to their values, not yours, getting along with them
will be much easier. Many proverbs are very old. So
some of the values they teach may not be as important in
the culture as they once were.

For example, Americans today do not pay much attention
to the proverb “Haste makes waste”, because patience is
not important to them. But if you know about past values, it
helps you to understand the present. And many of the older
values are still strong today.

Benjamin Franklin, a famous American diplomat, writer
and scientist, died in 1790, but his proverb “Time is money”
is taken more seriously by Americans of today than ever
before. A study of proverbs from around the world
shows that some values are shared by many cultures. In
many cases though, the same idea is expressed differently.
\subsection{申请大学最怕什么}
\begin{margintable}\vspace{-2cm}\footnotesize
	\includemedia[
	addresource=sound/fivestep/L24-申请大学注意事项-挑战原声.mp3]{\faBullhorn}{VPlayer.swf}
\end{margintable}
Parents and teachers will tell you not to worry when applying
for a place at university, but in the same breath will remind
you that it is the most important decision of your life.

The first decision is your choice of course. It will depend
on what you want to get out of university, what you
are good at and what you enjoy.

The next decision is where to apply. Aim high but within
reason. Do you have the right combination of subjects and
are your expected grades likely to meet entry requirements?

The deadline is January 15th. But it is best to submit your
application early because universities begin work as soon
as forms start rolling in.

The most important part of the application is the
much-feared personal statement. This is your chance to
convey boundless enthusiasm for the subject. So economy
of expression is foremost. Omit dull and ineffective generalities
and make sure you give concrete examples.

Admissions officers read every personal statement that
arrives. It is not convincing if you say you have chosen
the subject because you enjoy it. You have to get across
what it is about a particular area that has inspired you.
They will look for evidence that you have reflected and
thought about the subject.

Applicants should be honest. There is no point saying you
run marathons if you are going to be out of breath arriving
at the interview on the second floor.
\subsection{合作者,求带型,懒猪,独行侠}
\begin{margintable}\vspace{-2cm}\footnotesize
	\includemedia[
	addresource=sound/fivestep/L25-合作者-求带型-懒人-独行侠-挑战原声.mp3]{\faBullhorn}{VPlayer.swf}
\end{margintable}
At school and at work, I have noticed that people have
different kinds of work habits. Some people are collaborators
who like to work in groups. They find that doing a project
with someone else makes the job more pleasant and
the load lighter. Collaborators never work alone
unless they are forced to.

A second category I have noticed is the advice seeker.
An advice seeker does the bulk of her work alone but
frequently looks to others for advice. When this worker has
reached the crucial point in her project, she may show it
to a classmate or a co-worker just to get another opinion.
Getting the advice of others makes this worker feel secure
about her project as it takes shape.

Another type of worker I have noticed is the slacker.
A slacker tries to avoid work whenever possible. If he
seems to be busy at the computer, he is probably playing
a game online and if he is writing busily, he is probably
making his grocery list. Slackers will do anything except
the work they are paid to do.

The final type of worker is the loner. This type of worker
prefers working alone. This type of worker has confidence
in his ability and he is likely to feel that collaboration is a
waste of time. Loners work with others only when
they are forced to.

Collaborators, advice seekers, slackers and loners have
different work styles, but each knows the work habits that
help him or her to get the job done.

\subsection{今年过节不收礼,收礼就收……}
\begin{margintable}\vspace{-2cm}\footnotesize
	\includemedia[
	addresource=sound/fivestep/L26-今年过节不收礼-挑战原声.mp3]{\faBullhorn}{VPlayer.swf}
\end{margintable}
The topic of my talk today is gift-giving. Everybody likes to
receive gifts, right? So you may think that gift-giving is a
universal custom, but actually the rules of gift-giving vary
quite a lot. And not knowing them can result in great embarrassment.

In North America the rules are fairly simple. If you are invited
to someone's home for dinner, bring wine or flowers,
or a small item from your country. Among friends, family
and business associates, we generally don't give gifts
on other occasions except on someone's birthday and
Christmas.

The Japanese, on the other hand, give gifts quite frequently,
often to thank someone for their kindness. The tradition
of gift-giving in Japan is very ancient. There are many
detailed rules for everything, from the color of the wrapping
paper to the time of the gift presentation.

And while Europeans don't generally exchange business
gifts, they do follow some formal customs when visiting
homes, such as bringing flowers. The type and color of
flowers, however, can carry special meaning.

Today, we have seen some broad differences in gift-giving.
I could go on with additional examples, but let's not
miss the main point here. If we are not aware of and
sensitive to cultural differences, the possibilities for miscommunication
and conflict are enormous.

Whether we learn about these differences by reading a
book or by living abroad, our goal must be to respect differences
among people in order to get along successfully
with our global neighbors.

\subsection{头疼怎么办?}
\begin{margintable}\vspace{-2cm}\footnotesize
	\includemedia[
	addresource=sound/fivestep/L27-头疼怎么办-挑战原声.mp3]{\faBullhorn}{VPlayer.swf}
\end{margintable}
Almost everyone suffers from a headache occasionally.
But some people suffer from repeated, frequent headaches.
A headache is important because it can be
the first warning of a serious condition that could probably
be controlled if discovered early.

If a person removes the warning, day after day, with a
painkiller, he or she may pass the point of easy control.
The professional name for covering up a symptom is
“masking.” A headache specialist once said, “Masking
symptoms is not the best way of treatment. Sometimes
it is wiser to stand still than to advance in darkness.”

A headache often interferes needlessly with normal,
happy living. The employee with a headache does less
work. In a flash of temper, he or she may upset fellow
workers or customers, causing a direct or indirect loss to
the organization. The mother with a headache suffers and
disturbs the family. She upsets her husband and children.

Rest, quiet and fresh air stop many common headaches.
Lying down and possibly falling asleep may help.
One can often handle tension headaches by rubbing and
pressing neck muscles. Heat from an electric pad or a
warm bath can also help.

Because hunger may be overlooked as a headache
source, one must make a habit of regular meals. If a meal
must be postponed for more than an hour, a snack helps
to avoid a hunger headache.



\subsection{减灾的意义}
\begin{margintable}\vspace{-2cm}\footnotesize
	\includemedia[
	addresource=sound/fivestep/L28-减灾的意义-挑战原声.mp3]{\faBullhorn}{VPlayer.swf}
\end{margintable}
The Decade for Natural Disaster Reduction is a program
designed to reduce the impact of natural disasters
throughout the world.

With support from the United Nations, countries will be encouraged
to share information about disaster reduction.
For instance, information about how to plan for and cope
with hurricanes, earthquakes and other natural disasters.

One of the most important things the program plans to do
is to remind us of what we can do to protect ourselves. For
example, we can pack a suitcase with flashlights, a radio,
food, drinking water and some tools. This safety kit may
help us survive a disaster until help arrives.

Besides, the program will encourage governments to establish
building standards, emergency response plans,
and training programs, these measures can help to limit
the destruction by natural disasters.

The comparatively mild effects of the northern California
earthquake in 1989 are good evidence that we do
have the technology to prevent vast destruction.

The recent disasters, on the other hand, prove that people
will suffer if we don't use that technology. When a highway
collapsed in northern California, people were killed in their
cars. The highway was not built according to today's
strict standards to resist earthquakes.

Individuals and governments have to be far-sighted. We
should take extra time and spend extra money to build disaster
safety into our lives. Although such a program can't
hold back the winds or stop earthquakes, it can save people's
lives and homes.
\subsection{英国的监狱}
\begin{margintable}\vspace{-2cm}\footnotesize
	\includemedia[
	addresource=sound/fivestep/L29-英国的监狱-挑战原声.mp3]{\faBullhorn}{VPlayer.swf}
\end{margintable}
In Britain, if you are found guilty of a crime, you can be
sent to prison or be fined or be ordered to do community
work such as tidying public places and helping the old.

You may also be sent to special centers where you learn
practical skills like cooking, writing and car maintenance.
Around 5 percent of the prison population are women.

Many prisons were built over one hundred years ago. But
the government will have built 11 new prisons by next
year. There are two sorts of prisons: the open sort and the
closed sort.

In the closed sort, prisoners are given very little freedom.
They spend three to ten hours outside their cells when they
exercise, eat, study, learn skills, watch TV and talk to other
prisoners.

All prisoners are expected to work. Most of them are
paid for what they do, whether it is doing maintenance or
cooking and cleaning.


Prisoners in open prisons are locked up at night, but for
the rest of the time, they are free within the prison grounds.
They can exercise, have visitors, or study. And some
are allowed out of the grounds to study or do community
work.
\subsection{摇头 yes 点头 no}
\begin{margintable}\vspace{-2cm}\footnotesize
	\includemedia[
	addresource=sound/fivestep/L30-点头yes摇头no-挑战原声.mp3]{\faBullhorn}{VPlayer.swf}
\end{margintable}
Body language, especially gestures, varies among cultures.
For example, a nod of the head means “yes” to most
of us. But in Bulgaria and Greece a nod means “no”
and a shake of the head means “yes”.

Likewise, a sign for “OK”, forming a circle with our forefinger
and thumb, means zero in France and money in Japan.
Waving or pointing to an Arab business person
would be considered rude because that is how Arabs call
their dogs over. Folded arms signal pride in Finland, but
disrespect in Fiji\sn{斐济}.

The number of bows that the Japanese exchange on
greeting each other, as well as the length and the depth of
the bows, signals the social status each party feels towards
the other. Italians might think you're bored unless you use
a lot of gestures during discussions.

Many American men sit with their legs crossed with one
ankle resting over the opposite knee. However, this
would be considered an insult in Muslim countries, where
one would never show the sole of the foot to a guest.

Likewise, Americans consider eye contact very important,
often not trusting someone who is afraid to look you in the
eye. But in Japan and many Latin American countries,
keeping the eyes lowered is a sign of respect. To look a
partner full in the eye is considered a sign of ill-breeding
and is felt to be annoying.

\subsection{露营时,你睡在哪里?}
\begin{margintable}\vspace{-2cm}\footnotesize
	\includemedia[
	addresource=sound/fivestep/L31-露营时你睡在哪里-挑战原声.mp3]{\faBullhorn}{VPlayer.swf}
\end{margintable}
Today I'm going to talk about tents. Camping is still
one of the cheapest ways of having a holiday. And each
year, over 3 million people take camping vacations, either
here in Britain or abroad, mostly on the Continent.

Obviously, camping can't be as comfortable as living in a
permanent house, but modern tents can be very comfortable
indeed, with windows, bedrooms, kitchens and sitting
rooms.

The most popular tent sold in Britain is the frame
tent with two bedrooms and sleeping accommodation for
four people. There is usually an outer tent of water-
proofed fabric and a lighter inner tent or tents with a
built-in ground sheet.

The outer tent fits over the frame work. This is made of
metal poles which are fitted together. The inner tent is attached
to this frame. Generally, the inner tent is about half
the area of the outer tent. The other half of the outer tent is
the living area. This doesn't usually have a ground sheet
but you can buy one to fit, though it costs extra.

The ordinary four-bed frame tent doesn't usually have a
separate kitchen area, but the larger ones often do. 
You can buy a kitchen extension for many tents, and it's
worth buying one if you plan to stay camping in one place
for more than a few days.
\subsection{你的牙齿是活的}
\begin{margintable}\vspace{-2cm}\footnotesize
	\includemedia[
	addresource=sound/fivestep/L32-你的牙齿是活的-挑战原声.mp3]{\faBullhorn}{VPlayer.swf}
\end{margintable}
Take care of your teeth and your teeth will take care of
you. Your teeth are a living part of your body. They
have nerves and blood vessels. Diseased teeth can cause
pain, die and fall out.

Plaque is the main enemy of healthy teeth. Everyone
has plaque. It is a sticky, colorless film that coats the teeth.
Plaque is always forming on the teeth, especially at the
gum line.

If plaque is not removed, it builds up and gets under the
gum line. Plaque that is left on the teeth for some time
hardens. The result is tooth decay and gum disease.

The bacteria in plaque live on sugar. They change
sugar into acids, which break down the tooth's harder
outer covering. If left untreated, decay goes deeper and
deeper into the tooth.

After a while, the decay reaches the nerves and blood
vessels of the inner tooth. By the time this happens, the
tooth has probably started to ache.

In addition to tooth decay, there're also gum diseases
to watch out for. The bacteria and plaque make
poisons that attack the gums. Small pockets form around
the teeth. The pockets trap more bacteria and food particles.

Finally, the bone supporting the teeth is attacked and
starts to shrink. Teeth become loose and may fall out.
Adults lose most teeth this way. Keep your mouth
healthy. When you brush your teeth, do a good job.

\subsection{照顾老人不容易}
\begin{margintable}\vspace{-2cm}\footnotesize
	\includemedia[
	addresource=sound/fivestep/L33-照顾老人不容易-挑战原声.mp3]{\faBullhorn}{VPlayer.swf}
\end{margintable}
When people care for an elderly relative, they often do not
use available community services such as adult daycare
centers. If the caregivers are adult children, they are more
likely to use such services, especially because they
often have jobs and other responsibilities.

In contrast, a spouse, usually the wife, is much less likely
to use support services or to put the dependent person in
a nursing home. Social workers discover that the wife
normally tries to take care of her husband herself for as
long as she can in order not to use up their life savings.

Researchers have found that caring for the elderly can be
a very positive experience. The elderly appreciated the
care and attention they received. They were affectionate
and cooperative.

However, even when caregiving is satisfying, it is hard
work. Social workers and experts on aging offer caregivers
and potential caregivers help when arranging for the care
of an elderly relative.

One consideration is to ask parents what they want before
they become sick or dependent. Perhaps they prefer going
into a nursing home and can select one in advance.
On the other hand, they may want to live with their adult
children.

Caregivers must also learn to state their needs and
opinions clearly and ask for help from others, especially
brothers and sisters. Brothers and sisters are often willing to
help, but they may not know what to do.

\subsection{来听一段天气预报}
\begin{margintable}\vspace{-2cm}\footnotesize
	\includemedia[
	addresource=sound/fivestep/L34-一段天气预报-挑战原声.mp3]{\faBullhorn}{VPlayer.swf}
\end{margintable}
Now the weather forecast. It's a mixed picture over the
next few days. Today, very wet and windy in Northern Europe.

You can see from the satellite picture that the highest temperatures,
as they so often are, are in the southern
parts of Europe, where it's also quite dry, particularly over
the eastern parts of the Mediterranean.

The forecast suggests that it's going to be quite cold over
northwestern parts of Europe for the rest of the day, even
some snow on the Scandinavian mountains.

So that's today's weather, with showery conditions in many
parts of Northern Europe but the best of the sunshine in the
south and throughout the Mediterranean. And pretty
good but cool in the eastern parts of Europe, too.

Now let's look at tomorrow's weather chart. Very
much the same in the south except that the rain is starting
to push down into the northern parts of the Mediterranean
there.

Elsewhere, staying fine in Eastern Europe and fine in central
and eastern parts of the Mediterranean as well. But still
wet and windy in many northwestern parts of Europe, including
southern parts of Scandinavia, and a bit cool too.
\subsection{酗酒者并非故意}
\begin{margintable}\vspace{-2cm}\footnotesize
	\includemedia[
	addresource=sound/fivestep/L35-酗酒者并非故意-挑战原声.mp3]{\faBullhorn}{VPlayer.swf}
\end{margintable}
Alcoholism is a serious disease. Nearly nine million Americans
alone suffer from the illness. Many scientists disagree
about what the differences are between an alcohol addict
and a social drinker.

The difference occurs when someone needs to drink. And
this need gets in the way of his health or behavior. Alcohol
causes a loss of judgment and alertness. After a long
period, alcoholism can deteriorate the liver, the brain and
other parts of the body.

The illness is dangerous, because it is involved in half of all
automobile accidents. Another problem is that the
victim often denies being an alcohol addict and won't get
help. Solutions do exist. Without the assistance, the victim
can destroy his life.

All the causes of the sickness are not discovered yet.
There is no standard for a person with alcoholism. Victims
range in age, race, sex and background. Some groups of
people are more vulnerable to the illness.

People from broken homes and North American Indians
are two examples. People from broken homes often lack
stable lives. Indians likewise had the traditional life
taken from them by white settlers who often encouraged
them to consume alcohol to prevent them from fighting
back.

The problem has now been passed on. Alcoholism is
clearly present in society today. People have started to
get help and information. With proper assistance, victims
can put their lives together one day.
\subsection{道路千万条,安全第一条}
\begin{margintable}\vspace{-2cm}\footnotesize
	\includemedia[
	addresource=sound/fivestep/L36-道路千万条安全第一条-挑战原声.mp3]{\faBullhorn}{VPlayer.swf}
\end{margintable}
Wherever you go and for whatever reason, it's important to
be safe. While the majority of people you will meet when
travelling are sure to be friendly and welcoming,
there are dangers. Theft being the most common.

Just as in your home country, do not expect everyone you
meet to be friendly and helpful. It's important to prepare
for your trip in advance and to take precautions while you
are travelling.

As you prepare for your trip, make sure you have the
right paperwork. You don't want to get to your destination,
only to find you have the wrong visa, or worse, that your
passport isn't valid anymore. Also, make sure you travel
with proper medical insurance, so that if you are sick or injured
during your travels, you will be able to get treatment.

If you want to drive while you are abroad, make sure you
have an international driver's license. When you get
to your destination, use official transport. Always go to bus
and taxi stands. Don't accept rides from strangers who
offer you a lift. If there is no meter in the taxi, agree on a
price before you get in.

If you prefer to stay in cheap hotels while travelling, make
sure you can lock the door of your room from the inside.
Finally, remember to smile. It's the friendliest and most sincere
form of communication, and is sure to be understood
in any part of the world!




\subsection{你的对手来自全世界}
\begin{margintable}\vspace{-2cm}\footnotesize
	\includemedia[
	addresource=sound/fivestep/L37-你的对手来自全世界-挑战原声.mp3]{\faBullhorn}{VPlayer.swf}
\end{margintable}
One of the biggest challenges facing employers and
educators today is the rapid advance of globalization. The
market place is no longer national or regional, but extends
to all corners of the world. And this requires a global-ready
workforce.

Universities have a large part to play in preparing students
for the 21st century labor market by promoting international
educational experiences.

The most obvious way universities can help develop a
global workforce is by encouraging students to study
abroad as part of their course. Students who have 
experienced another culture first-hand are more likely to be
global-ready when they graduate.

It is important to point out that students also need to
have a deep understanding of their own culture before
they can begin to observe, analyze and evaluate other
cultures.

In multicultural societies, people can study each other's
cultures to develop intercultural competencies, such as
critical and reflective thinking and intellectual flexibility.
Many universities are already embracing this challenge
and providing opportunities for students to become global
citizens.

Students themselves, however, may not realize that
when they graduate, they will be competing in a global
labor market. And universities need to raise awareness of
these issues amongst undergraduates.
\subsection{保健因素 VS 激励因素}
\begin{margintable}\vspace{-2cm}\footnotesize
	\includemedia[
	addresource=sound/fivestep/L38-保健因素vs激励因素-挑战原声.mp3]{\faBullhorn}{VPlayer.swf}
\end{margintable}
It's logical to suppose that things like good labor relations,
good working conditions, good wages and benefits and
job security motivate workers, but one expert, Fredrick
Herzberg argued that such conditions do not motivate
workers. They are merely satisfiers\sn{保健因素}.

Motivators\sn{激励因素}, in contrast, include things
such as having a challenging and interesting job, recognition
and responsibility.

However, even with the development of computers
and robotics, there are always plenty of boring, repetitive
and mechanical jobs, and lots of unskilled people who
have to do them.

So how do mangers motivate people in such jobs? One
solution is to give them some responsibilities, not as individuals,
but as part of a team.

For example, some supermarkets combine office
staff, the people who fill the shelves, and the people who
work at the checkout into a team, and let them decide
what product lines to stock, how to display them, and so
on .

Many people now talk about the importance of a company's
shared values or culture, with which all the staff
can identify: for example, being the best hotel chain, or
making the best, the most user-friendly or the most reliable
products in a particular field.
\subsection{欧罗巴,木卫二}
\begin{margintable}\vspace{-2cm}\footnotesize
	\includemedia[
	addresource=sound/fivestep/L39-欧罗巴-木卫二-挑战原声.mp3]{\faBullhorn}{VPlayer.swf}
\end{margintable}
Since early times, people have been fascinated with the
idea of life existing somewhere else besides earth.
Until recently, scientists believed that life on other planets
was just a hopeful dream.

But now they are beginning to locate places where life
could form. In 1997, they saw evidence of planets near
other stars like the sun. But scientists now think that life
could be even nearer in our own solar system.

One place scientists are studying very closely is Europa,
a moon of Jupiter. Space probes have provided
evidence that Europa has a large ocean under its surface.
The probes have also made scientists think that under its
surface Europa has a rocky core giving off volcanic heat.

Water and heat from volcanic activity are two basic
conditions needed for life to form. A third is certain basic
chemicals such as carbon, oxygen and nitrogen. Scientists
believe there might be such chemicals lying at the bottom
of Europa's ocean. They may have already created life or
may be about to.

You may wonder if light is also needed for life to form.
Until recently, scientists thought that light was essential.
But now, places have been found on earth that are in total
blackness such as caves several miles beneath the surface.
And bacteria, primitive forms of life, have been seen
there.

So the lack of light in Europa's sub-surface ocean doesn't
automatically rule out life forming.
\subsection{全球失业潮,老板压力小?}
\begin{margintable}\vspace{-2cm}\footnotesize
	\includemedia[
	addresource=sound/fivestep/L40-全球失业潮老板压力小-挑战原声.mp3]{\faBullhorn}{VPlayer.swf}
\end{margintable}
A recent International Labor Organization report says
the deterioration of real wages around the world calls into
question the true extent of an economic recovery, especially
if government rescue packages are phased out too
early. The report warns the picture on wages is likely to
get worse this year despite indications of an economic rebound.

Patrick Belser, an International Labor Organization specialist,
says declining wage rates are linked to the levels of
unemployment.

“The quite dramatic unemployment figures, which we now
see in some of the countries strongly suggest that there will
be greater pressure on wages in the future as more people
will be unemployed, more people will be looking
for jobs and the pressure on employers to raise wages to
attract workers will decline. So, we expect that the second
part of the year will not be very good in terms of wage
growth.”

The report finds more than a quarter of the countries experienced
flat or falling monthly wages in real terms. They
include the United States, Austria, Costa Rica, South Africa
and Germany. International Labor Organization economists
say some nations have come up with policies to lessen
the impact of lower wages during the economic crisis.

An example of these is Work Sharing with government
subsidies. Under this scheme, the number of individual
working hours is reduced in an effort to avoid layoffs. For
this scheme to work, the government must provide wage
subsidies to compensate for lost pay due to the shorter
hours.